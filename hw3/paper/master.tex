%Doc config

\documentclass[11pt,letterpaper]{article}

\usepackage{graphicx}
\usepackage{textcmds}
\usepackage[left=3cm,right=3cm,top=3cm,bottom=3.8cm]{geometry}
\usepackage{tabularx}
\newcolumntype{b}{X}
\newcolumntype{s}{>{\hsize=.45\hsize}X}
\newcolumntype{r}{>{\hsize=.35\hsize}X}
\newcolumntype{t}{>{\hsize=.2\hsize}X}
\usepackage{booktabs}
\usepackage{float}
\usepackage{hyperref}
\usepackage{comment}

\usepackage[
    backend=biber,
    style=apa,
    sortlocale=de_DE,
    natbib=true,
    url=false,
    doi=true,
    eprint=false
]{biblatex}

\usepackage{caption}




%\addbibresource{biblio.bib}

\newcommand{\fontsmall}{\fontsize{7pt}{10pt}\selectfont}
\newcommand{\fontnormal}{\fontsize{11pt}{16pt}\selectfont}
\newcommand{\headline}{\fontsize{17pt}{26pt}\selectfont}

\bibliography{biblio}


\begin{document}


   \title{Personal Reflection: Online Social Networks}



\noindent\begin{minipage}{0.5\textwidth}
\fontsmall
Management Center Innsbruck \\
Department Management, Communication \& IT  \\
ILV Digital Behavior


\end{minipage}%
\hfill%
\begin{minipage}{0.3\textwidth}\raggedleft
\includegraphics[height=1.0cm]{mci-logo.png}

\end{minipage}



\begin{center}
\textbf{Homework: Digital Behavior}\\   %Title
Florian Bigelmaier\\                         %Name
December 9th, 2021\\                         %Date
\end{center}
\rule{\linewidth}{0.1mm}




\headline \begin{center}
Personal Reflection on My Social Network Online
\end{center}
\fontnormal

\section*{Description of Social Network Use}
\begin{center}
\textit{\qq{90\% of young adults use social media, compared with 12\% in 2005}}\autocite[][]{Perrin.2015}
\end{center}

Online social media matured in only ten years from a niche service to an essential part of youth culture and life. There are obviously reasons behind this enormous adoption and behavior. But contrarily, Social Media seams also to have its shady sides - a discussion that becomes currently more and more a mainstream issue, as the \textit{Lush} case recently showed \autocite[][]{Wood.2021}. This paper may reflect, in which way I use social media. I focused on \textit{Instagram}, my most used social media, though it does not represent my complete online social behavior.

Instagram appeals to me much because of several reasons. First, it allows me to build up a strong network of various people and stay in contact with them, support others and keep track of former fellow students or classmates. A major advantage is, that all my friends are using Instagram. The management of weak ties does also play a big role for me by using features such as stories and messages as well as a contact to strong ties. This is also an important reason for me to stick to the platform. I will analyze this point in part II in detail. Additionally, Instagram is for me an important source of entertainment and a good opportunity to keep up to date regarding changes e.g., in my native home region.

It is difficult to answer the question, what my friends' arguments are to choose a social network - at least they also have chosen Instagram as one for their portfolio and usually it is one of the most important ones to them: In the circle of people I count as friends, everyone uses Instagram actively. Nevertheless, other people that have strong ties to me do not rely on this social network, such as my parents. They refuse to use it (completely or actively) because of missing strong ties on the platform, which means less fear of missing out and less possibility to consume personal related content. Slower technology adoption does also play a role as much as the fact, that for my parents, most of their friends are in a very local surrounding of a few kilometers. So, there is no essential need to substitute physical contact.

Instagram has evolved in steps since it has established in 2010. It started with the feature of posting images combined with filters, later new features joined and the application in total became more of a classical social network. As a user, I do not desire any new features. Much more, I would propose to change Instagram into a more sustainable and responsible platform. Therefore, I propose two adaptions for the future.

The Norwegian legislature recently passed a law forcing professional users of social media sites to be transparent about editing or manipulating images \autocite[][]{PressReynolds.2021}. This should prevail young people from comparing them with unattainable ideals of beauty \autocite[][]{PressReynolds.2021}. The BBC referred to a study that exposes, that 95\% of the youths nowadays \qq{consider changing their appearance by doing things like dieting or getting surgery} \autocite[][]{Grant.2021}, this does lead to dissatisfaction \autocites[p. 501]{Babaleye.2020}[p. 1]{Rafati.2021}. In my opinion this amount is definitely too high and it is significant for youths health to ensure that Instagram applies the new Norwegian restrictions in the whole world.

A second issue I want to refer to is content moderation. With the beginning of internet platforms, crime also found its way to misuse these platforms \autocite[][]{Gupta.2021}. Moreover, content that might not be suitable for certain members of social networks (such as pictures that show violence) \autocite[][pp. 225 - 226]{Mengu.2015}, copyright infringements \autocite[][]{Hegemann.2019} and spam \autocite[][p. 1458]{Jin.2011} are topics Social Media companies have to deal with. Only limited commitment is spent on solving these issues. Content moderation might be one solution, but it is closely connected with censorship and contradicts the thought of free speech \autocite[][p. 1359]{Langvardt.2017}. \cite{Langvardt.2017} refers to this indissoluble conflict as \qq{the dilemma of the moderators}. Pure automation will not solve the issue \autocite[][pp. 3 - 4]{Gillespie.2020}. Though there might never be a perfect solution for this difficult topic, in my opinion it is crucial that social media companies spend more time and effort within projects and investigation on how to solve these problems. 


\section*{Quantification of Social Media Use}
In the second part of this paper, I will quantify my personal Social Network use. For this purpose, I will analyze my visible interaction within the time frame from September 1st until November 30th, 2021. In this time frame, I posted one image to my main feed, I shared 32 pictures within the story or the close-friends-story feature, I pressed 108 times the like button, chatted with 36 different persons, from whom I would count 6 as strong ties and placed one comment. I only submitted six comments in total since I have registered in 2013. Regarding the messages there should be a further differentiation: Within 13 chats, the conversation consists only out of one to three single messages that often refer to an Instagram story or a content of a third user, that has been shared. On average, I consume 77 posts per day \autocite[whole paragraph:][]{Instagram.2021}. 

What might be derived of this data is, that on the one hand, I use Instagram heavily to entertain myself. Personal communication in the sense of reciprocal activity does take place, but does not count for as much time as the singular activity. Furthermore, it is interesting, that comments do not matter for my personal use of Instagram, whereas I use the messaging feature regularly. With strong ties, the communication is much more reciprocal: conversations last longer and have a deeper emotional aspect, whereas communication with weak ties usually consists of short reactions on their activities.

Regarding the social capital on Instagram, one can differentiate between bonding, bridging and maintaining. While Instagram helps me to keep up bonds with strong ties in the sense that emotional experiences can be shared, it also enhances the possibility to have bridges to certain communities, such as community of my native home village. Though, it has to be mentioned that the effect of bridging does way more apply on LinkedIn for myself. Maintaining relationships is one of the main reasons to stay on the platform: As already stated, lots of my friends and acquaintances are using this platform and therefore it is an easily usable tool to keep track of what is important in their lives. \newline


\noindent \textbf{Words:} 1064

\newpage

\printbibliography

\newpage






\end{document}
