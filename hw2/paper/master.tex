%Doc config

\documentclass[11pt,letterpaper]{article}

\usepackage{graphicx}
\usepackage{textcmds}
\usepackage[left=3cm,right=3cm,top=3cm,bottom=3.8cm]{geometry}
\usepackage{tabularx}
\newcolumntype{b}{X}
\newcolumntype{s}{>{\hsize=.45\hsize}X}
\newcolumntype{r}{>{\hsize=.35\hsize}X}
\newcolumntype{t}{>{\hsize=.2\hsize}X}
\usepackage{booktabs}
\usepackage{float}
\usepackage{hyperref}


\usepackage[
    backend=biber,
    style=apa,
    sortlocale=de_DE,
    natbib=true,
    url=false,
    doi=true,
    eprint=false
]{biblatex}

\usepackage{caption}




%\addbibresource{biblio.bib}

\newcommand{\fontsmall}{\fontsize{7pt}{10pt}\selectfont}
\newcommand{\fontnormal}{\fontsize{11pt}{16pt}\selectfont}
\newcommand{\headline}{\fontsize{17pt}{26pt}\selectfont}

\bibliography{biblio}


\begin{document}


   \title{Personal Reflection: Information Behavior}



\noindent\begin{minipage}{0.5\textwidth}
\fontsmall
Management Center Innsbruck \\
Department Management, Communication \& IT  \\
ILV Digital Behavior


\end{minipage}%
\hfill%
\begin{minipage}{0.3\textwidth}\raggedleft
\includegraphics[height=1.0cm]{mci-logo.png}

\end{minipage}



\begin{center}
\textbf{Homework: Digital Behavior}\\   %Title
Florian Bigelmaier\\                         %Name
December 2nd, 2021\\                         %Date
\end{center}
\rule{\linewidth}{0.1mm}




\headline \begin{center}
Personal Reflection on Information Behavior
\end{center}
\fontnormal

\section*{Personal Experience: Observing on Information Behavior}
In the recent 50 years, the share of the service sector of the Austrian economy grew strongly \autocite[][]{StatistikAustria.2021}. Consequently, young people might enter the labor market more often in jobs as \textit{knowledge workers} todays. Adequate information behavior is a major skill for such positions. Beneath that, fake news and uncertainty regarding authenticity and integrity of information in the Web 2.0 are topics that came up in the last two decades. Because of these factors, it might be interesting to analyze the young generations information behavior, like \cite{Nicholas.2011} did.

For a personal reflection, I want to combine the results of Nicholas' et aliorum paper with a personal task: I wanted to know about the function of mRNA vaccines as I had basic knowledge until then only about the efficacy and safety. For further explanation of the task, see the appendix. The task appears to be like one example of \cite[p. 31-33]{Case.2007}, who refers to his example as a search out of \qq{curiosity} with a \qq{visceral need}. I tried to find out more information for this task via Internet research as this seemed to bring the highest efficiency and the needed topicality for the question. On the downside, it is hard to find reliable material in the information flood of the internet. The main ambition was not about finding a certain amount of information. Much more, I wanted to expand my knowledge with reliable information, so information quality was the goal, the quantity was only a secondary aim.

The task has been observed by me at the time of execution in a diary format. It describes which obvious actions have been taken. Therefore, it is a summary of what information was perceived and which sources were used. It also depicts (in combination with the time used for reading and watching) how deeply information has been recognized. On the other hand, it is not possible to make any conclusions about what has been read and integrated in the observed persons mind. It seems to depend on the question of research, how well the diary reflects the information behavior. In this experiment there were several points in the internal processes that might differ from the observed behavior: Jumping over information has been conducted often, the reasons are not always obvious or as one might think. I also encountered moments when I thought of changing the search string or search engine. Only the results can be seen within an entry of a new search. If I would repeat this assignment freely, I would also take more personal notes on what happened internally to compare the inner and outer perspective better.

\section*{Being Part of the Google Generation - Personal Reflection}
To bring the observations of the diary together with \cite{Nicholas.2011} paper, I want to focus on three points. First, the authors found out, that persons that belong to the generation who raised with Google, need less page visits to answer a certain question on average \autocite[p. 41]{Nicholas.2011}. Therefore, \cite[p. 44]{Nicholas.2011} allege the younger generation a lower ambition on information quality. In the diary there are proves and contradictions to this: On the one hand, it would have been possible to view more sites. On the other hand, selecting carefully a rather small number of sites by their quality refutes their interpretation. If this might also apply to the whole generation is a question which is not researched by the paper.

Second, \cite[pp. 31-32]{Nicholas.2011} analyzed the information behavior of several participants and came up with metaphors of animals that might have certain characteristics  in common with certain types of information seekers. For the researched question in this paper, the behavior might be a mixture of the \qq{web hedgehog} that concentrates on one question at a time and rather takes his time than going with bad information \autocite[p. 31]{Nicholas.2011}. Also, some of the characteristics  of the \qq{web octopus} apply on the recorded behavior: Once I found out, that a normal Ecosia search might not reach, I switched to more sophisticated sources and search engines \autocite[p. 32]{Nicholas.2011}. Social networks did not play a role at all in my information behavior.

A third interesting point focuses multi-tasking. \cite[p. 38]{Nicholas.2011} state, that the Generation I belong to (i.e., \qq{the Google Generation}), would not be able to conduct multi-tasking as well as Generation Y, but might apply multi-tasking most, which was the result of a cited Ofcam report. For myself, I left any further tasks aside (such as listening to music, instant messaging etc.) as I usually do, when I need focused attention. Furthermore, I concentrated on one piece of information at a time. I do not doubt the results, but I am not able to find this behavior in my observations.

In conclusion, the paper of \cite[pp. 31-32]{Nicholas.2011} provides several interesting insights, but also interprets subjectively information that is not based on the conducted experiment. Some of these statements in the conclusion and right in the results chapter are even formulated polemically. For me individually, I cannot observe the majority of these interpretations. To face these shortcomings, it might make sense to provide further empirical evidence and find causality and reasons to lift the discussion on a more pragmatic level.
\newline

\noindent \textbf{Words:} 868

\printbibliography

\newpage

\section*{Appendix: Diary}
\begin{enumerate}
\item \textit{Describe your task in detail}: Find out, what mRNA-vaccination is about, how the basic function behind it works. Use digital information systems for these tasks and look for any information that is good enough to fulfill the basic interest. The reliability of the information should be as high as possible.
\item \textit{Describe the situational factors affecting the task}: I already had a foundation of knowledge to this topic, especially regarding the efficacy and possible side effects. Furthermore, a few days before the task I received my third dose of mRNA vaccines. There were no further situational factors that influenced the execution of the task such as multi-tasking, music or noises, other people. instant messaging etc.
\item \textit{What is the ambition level you aim at in the task:  good, nearly good, or satisfactory?} The ambition regarding the reliability of the information is high and in the focus. The ambition level regarding the amount of information or gathered knowledge aims on a satisfactory level. Higher ambition might result if there is a lot of easily accessible reliable information.
\item \textit{Describe in detail what kind of information you think you need to perform the task} \newline
\textit{a. thoughts in the beginning of the task}: Basic introduction to the way mRNA vaccination is working (and the underlying biological facts). Analogies/ simplifications for the complex processes in the human organism might be necessary.\newline
\textit{b. thoughts emerged later during the task}: Later I thought, more detailed information might be needed regarding the biological processes, the rest stayed identically to the beginning.
\item \textit{Which channels and sources do you consider (mention also those you won't use)?} \newline
\textit{a) thoughts in the beginning of the task:} I will first consider trustworthy publication sites such as German/ Austrian public service broadcasting and long established, politically centered publishing houses. I will exclude any information that is published by sites that are either unknown by me or known, but are not trustworthy enough, such as publications of the Axel Springer house, Kronenzeitung or sites that are part of the Murdoch-empire and many more sites that are not independent enough or do not follow journalistic standards and ethics consequently. \newline
\textit{b) thoughts emerged later during the task:} Deeper information might be presented better by certain papers and academic publications.
\item \textit{How much time did you use in planning this information seeking?} Around 20 minutes.
\item \textit{Which channels and sources did you use? (Include yourself; mention the names of
any colleagues consulted; mention channels used no matter whether or not you obtained the sources):}
I focused on online research. I used the sources listed in the following table. The columns \qq{channel} including the justification for this will not be displayed as proposed in the template as the channel is for all information online. Online information has been chosen in this case for the fastest and most efficient way of finding information as well as for the sheer amount of available information and high-quality presentation of content, the topicality of the research question is also best covered in a fast-pace medium. The disadvantage of the internet is its high variety of content quality. Information seeker needs to filter out invalidated, unreliable, or badly prepared information. In the following tables the single used sources will be displayed.

\begin{table}[H]
	\centering

\begin{tabularx}{\textwidth}{bbss}
Source & Why chosen & Success & Applicability    \\ \toprule[2pt]
	{ORF (Austrian public service broadcasting)} & {Reliable information (highly reputed), ideal for basic information because it is aiming on the whole population} & (b) & (b)  \\ \midrule[0.5pt]
	{ZDF (German public service broadcasting)} & {Also, for reliable information (also highly reputed), also ideal for basic information because it is aiming on the whole population)} & (b) & (c)  \\ \midrule[0.5pt]
	{Augsburger Allgemeine (German newspaper with a focus on local, state, and countrywide news, conservative orientation)} & {the first media brand I personally raised with and built trust to; the information seemed until now reliable} & (c) & (c)  \\ \midrule[0.5pt]
	{Correctiv (renowned fact checking site with high degree of correct information)} & {Trustworthy site with several information, I had so far, a good experience with this site and no bigger unexplainable contradictions with the majority of other sources} & (b) & (b)  \\ \midrule[0.5pt]
	{Nature Journal (famous journal)} & {Famous journal and web site for scientific publishing and information} & (c) & (c)  \\ \midrule[0.5pt]
	{US National Library of Medicine at National Institutes of Health} & {public US American government institution that seems to be trustworthy} & (b) & (b)  \\ \midrule[0.5pt]
	{Annual review of medicine} & {Appeared to be trustworthy, but opened in good faith, but as it is unknown with distance} & (c) & (c)  \\ \midrule[0.5pt]
	{Publication Server of University of Ottawa} & {Public Canadian University,  opened in good faith, but (as the researchers were unknown for me) with distance} & (a) & (a)   \\ \bottomrule[1.3pt]
\end{tabularx}
\captionof{table}{\textbf{Used sources and channels}}

\end{table}

\fontsmall
Success: you got the information (a) wholly, (b) partly, (c) not at all \newline
Applicability: the information was (a) well-applicable, (b) partially applicable, (c)  not applicable at
all
\fontnormal


\item \textit{Was the whole of the information obtained sufficient for the task or insufficient for the task?} As there is no limit for getting in touch with the question of interest that has been the basis for this observation, it is difficult to state whether enough information has been found. Nevertheless, there have been found several information and the knowledge is broader afterwards.

\item \textit{Estimate the time spent on information seeking on the whole task.}
The whole task including the preparation for seeking a topic and developing a strategy took a bit less than one hour.

\item \textit{Recording of the information activity.}
In this paragraph the actual observation regarding the information task will be presented.
For this purpose, a coding scheme seems to be appropriate to structure the task. The coding scheme is deductive i.e., it has been determined in advance based on the usual information behavior.\\

Coding scheme: \\
\textit{C}: Observed person clicks on a link \\
\textit{C\textsubscript{c}}: Click + CTRL / Observed person clicks on a link while holding the Control key (Open a link in a new tab to prepare sites for later reading) \\
\textit{URL}: Open new tab, start typing in URL/ search bar \\
\textit{T}: Change tab \\
\textit{T\textsubscript{close}}: Close active tab, switch to another one \\
\textit{S\textsubscript{ecosia}}: Using a search engine, in the index the used search engine, here: Ecosia \\
\textit{S\textsubscript{onsite}}: Onsite search on a web site \\
For each used source there will be short code to make the table clearer e.g., \textit{ZDF2021}. Beneath the table a reference is shown that links directly to the used sites. \\

Furthermore, there should be the following abbreviation in the upcoming table for Search Engine: SE, as it is used frequently in the observation.

The observation task started on November 28th at 6:03 p.m. The time will be displayed in the first column relative in minutes to the starting time.

\begin{table}[H]
	\centering

\begin{tabularx}{\textwidth}{trb}
t & Coding scheme & Further explanation \& Observation \\ \toprule[2pt]
	{0 minutes after start} & {S\textsubscript{ecosia}} & {Opens Firefox Browser and uses standard SE, searches for \qq{mRNA Impfstoffe}, does not click on any result on the first page, clicks on the link for SE result page 2}  \\ \midrule[0.5pt]
	{ } & {C\textsubscript{c}} & {ORF2021, ZDF2021, AZ2021, COR2021}  \\ \midrule[0.5pt]
	{5} & {T} & {ORF2021, reads content of the page}  \\ \midrule[0.5pt]
	{7} & {T\textsubscript{close}} & {ZDF2021, reads first part of article, watches 2-minute video, reads second part}  \\ \midrule[0.5pt]
	{12} & {T\textsubscript{close}} & {AZ2021, scrolls over article and headlines, watches short video}  \\ \midrule[0.5pt]
	{15} & {T\textsubscript{close}} & {COR2021, reads, scrolls over big parts, holds on at headline about genetic technology}  \\ \midrule[0.5pt]
	{17} & {T\textsubscript{close} + S\textsubscript{ecosia}} & {Searches \textit{Nature Journal}}  \\ \midrule[0.5pt]
	{ } & {C} & {Clicks on search result Nature.com}  \\ \midrule[0.5pt]
	{18} & {S\textsubscript{onsite}} & {Searches on \href{https://www.nature.com}{Nature.com} for \textit{mRNA}}  \\ \midrule[0.5pt]
	{ } & { } & {scrolls over results}  \\ \midrule[0.5pt]
	{ } & {C} & {Stoye2021, scroll to mRNA vaccination section, reads paragraph}  \\ \midrule[0.5pt]
	{20} & {S\textsubscript{onsite}} & {Searches on Nature.com for \textit{mRNA function}}  \\ \midrule[0.5pt]
	{ } & { } & {scrolls over results}  \\ \midrule[0.5pt]
	{ } & {S\textsubscript{onsite}} & {Searches on Nature.com for \textit{mRNA explanation}}  \\ \midrule[0.5pt]
	{ } & { } & {scrolls over results, reads headlines}  \\ \midrule[0.5pt]
	{23} & {URL} & {\href{https://scholar.google.at}{Google Scholar}}  \\ \midrule[0.5pt]
    { } & {S\textsubscript{g.scholar}} & {searches for \textit{explanation mRNA vaccines}}  \\ \midrule[0.5pt]
	{ } & {C\textsubscript{c}} & {Hendaus2021, Samson2021, Hogan2021, Xia2021}  \\ \midrule[0.5pt]
  {26} & {T} & {Hendaus2021, reads slowly the whole article, constantly scrolling and reading}  \\ \midrule[0.5pt]
  {37} & {T} & {Samson2021, scrolls over page fast}  \\ \midrule[0.5pt]
	{38} & {T\textsubscript{close}, C} & {Hogan2021, clicks on PDF full text, scrolls to future outlook, skims text}  \\ \midrule[0.5pt]
	{43} & {T\textsubscript{close}, C} & {Xia2021, clicks on PDF full text, scrolls to paragraph 2.2, reads paragraph, scrolls to conclusion, reads }  \\ \midrule[0.5pt]
	{47} & { } & {Closes browser}  \\ \bottomrule[1.3pt]

\end{tabularx}
\captionof{table}{\textbf{Observation of information behavior}}

\end{table}

Used Sources:
\begin{itemize}
\item \href{https://mdpi-res.com/d_attachment/vaccines/vaccines-09-00734/article_deploy/vaccines-09-00734-v3.pdf}{Xia2021 (https://mdpi-res.com/)}
\item \href{https://www.annualreviews.org/doi/pdf/10.1146/annurev-med-042420-112725}{Hogan2021 (https://www.annualreviews.org/)}
\item \href{https://www.ncbi.nlm.nih.gov/pmc/articles/PMC7893482/}{Hendaus2021 (https://www.ncbi.nlm.nih.gov/)}
\item \href{https://www.chemistryworld.com/features/mrna-vaccines-for-covid-and-beyond/4014420.article}{Samson2021 (https://www.chemistryworld.com/)}
\item \href{https://www.nature.com/articles/d41586-021-02945-1}{Stoye2021 (https://www.nature.com/)}
\item \href{https://www.augsburger-allgemeine.de/panorama/Totimpfstoff-mRNA-Impfstoff-Vektorimpfstoff-Unterschiede-erklaert-id60881746.html}{AZ2021 (https://www.augsburger-allgemeine.de/)}
\item \href{https://correctiv.org/faktencheck/2021/01/22/mrna-impfstoffe-basieren-zwar-auf-gentechnik-aber-sind-keine-genmanipulation/}{COR2021 (https://correctiv.org/)}
\item \href{https://www.zdf.de/nachrichten/politik/corona-impfstoff-mrna-moderna-biontech-100.html}{ZDF2021 (https://www.zdf.de/)}
\item \href{https://kaernten.orf.at/stories/3087266/}{ORF2021 (https://kaernten.orf.at/)}

\end{itemize}



\end{enumerate}









\end{document}
