%Doc config

\documentclass[11pt,letterpaper]{article}

\usepackage{graphicx}
\usepackage{textcmds}
\usepackage[left=3cm,right=3cm,top=3cm,bottom=3.8cm]{geometry}
\usepackage{tabularx}
\newcolumntype{b}{X}
\newcolumntype{s}{>{\hsize=.45\hsize}X}
\newcolumntype{r}{>{\hsize=.35\hsize}X}
\newcolumntype{t}{>{\hsize=.2\hsize}X}
\usepackage{booktabs}
\usepackage{float}
\usepackage{hyperref}
\usepackage{comment}

\usepackage[
    backend=biber,
    style=apa,
    sortlocale=de_DE,
    natbib=true,
    url=false,
    doi=true,
    eprint=false
]{biblatex}

\usepackage{caption}




%\addbibresource{biblio.bib}

\newcommand{\fontsmall}{\fontsize{7pt}{10pt}\selectfont}
\newcommand{\fontnormal}{\fontsize{11pt}{16pt}\selectfont}
\newcommand{\headline}{\fontsize{17pt}{26pt}\selectfont}

\bibliography{biblio}


\begin{document}


   \title{Personal Reflection: Online Social Networks}



\noindent\begin{minipage}{0.5\textwidth}
\fontsmall
Management Center Innsbruck \\
Department Management, Communication \& IT  \\
ILV Digital Behavior


\end{minipage}%
\hfill%
\begin{minipage}{0.3\textwidth}\raggedleft
\includegraphics[height=1.0cm]{mci-logo.png}

\end{minipage}



\begin{center}
\textbf{Homework: Digital Behavior}\\   %Title
Florian Bigelmaier\\                         %Name
December 10th, 2021\\                         %Date
\end{center}
\rule{\linewidth}{0.1mm}




\headline \begin{center}
Personal Reflection on My Social Network Online
\end{center}
\fontnormal

\section*{Description of Social Network Use}
\begin{center}
\qq{Today, 90\% of young adults use social media, compared with 12\% in 2005, a 78-percentage point increase.} \autocite[][]{Perrin.2015}
\end{center}

Online social media matured in only ten years from a niche service to an essential part of youth culture and life with enormous impact on everyday life, communication, social relationships and communities, not even to mention the society in total. Obviously there are reasons behind this enormous adoption and behavior. On the other hand, Social Media seams also to have its shadow sides - criticism regarding the way social media is working even reached the marketing of popular brands: Lush, a UK soap manufacturer, decided to shut down their social media accounts and not support the big sites anymore \autocite[][]{Wood.2021}. This paper may reflect, in which way I use social media. To keep it short, I focused on \textit{Instagram}, my most used social media sites, though it does not represent my complete online social behavior ideally due to completely different behavior e.g. on networks or apps like \textit{LinkedIn} and \textit{Signal}.

Instagram appeals to me much because of several reasons that have all been discussed in lectures of this course. First of all, it allows me to build up a strong network of different people and stay in contact with them, support others and keep track of former fellow students or class mates. A major advantage is, that almost all my friends are using Instagram. The management of weak ties does also play a big role for me using Instagram by using features such as stories and messages. The management of strong ties plays a subordinated role, but also takes place on Instagram, e.g. by features such as messages and stories for close friends (stories that can only be seen by a narrower, determined circle of contacts). Messages with weak ties are inferior to those with strong ties in the length of the message, the depth of emotional connection, the frequency of receiving and sending and reply speed. To maintain weak ties is important to me as I have several friends whom I cannot be with due to physical distances, nevertheless they are important to me in their specific way. Just to pick one aspect: Instagram is for me an important source of entertainment and a good opportunity to keep up to date regarding changes e.g. in my native home region via strong and weak ties, that means good friends and family as well as more loose friends.

It is difficult to answer the question, what my friends arguments are to chose a social network - at least they also have chosen Instagram as one for their portfolio and usually it is one of the most important ones to them: In the circle of people I would count as friends, everyone uses Instagram actively. Nevertheless, other people that have strong ties to me do not rely on this social network, such as my parents. They refuse to use it (or at least actively) because of a missing network effect, which means less fear of missing out and less possibility to keep track on their real life social ties. Technology adaption does also play a role as much as the fact, that for my parents, most of their friends are in a very local surrounding of a few kilometers. The most used medium beneath meeting personally is the telephone.

Instagram has evolved in different steps since it started in 2010. It started with the usual feature of posting images combined with filters, later on it new features joined and the application in total became more like a social network with less focus on the single posted images. Regarding the feature set, it seams to me, that I do not desire any new features. Much more, I would propose to shift the Social Network into a more sustainable and responsible platform. Therefore, two changes might make sense.

Norway has been in the press in the last months for establishing a law, that forces professional users of Social Media sites to make transparent if they edited or manipulated images of their selves \autocite[][]{PressReynolds.2021}. This should prevail young people from comparing them with the appearance of persons who do not exist in this form and generate unreachable ideals of beauty \autocite[][]{PressReynolds.2021}. BBC referred to a study that exposes, that 95\% of the youths nowadays \qq{consider changing their appearance by doing things like dieting or getting surgery} \autocite[][]{Grant.2021}. In my opinion this amount is definitely to high and it is crucial for the health of the upcoming generations of youths to ensure, that Instagram applies the new Norwegian restrictions also in every other region of the world to protect younger users and maintain mental health. My argumentation might be supported by investigations like \cite[p. 501]{Babaleye.2020} who found out, that Instagram does affect the beauty ideals of female minors as well as that this \qq{can eventually lead to dissatisfaction and oftentimes measures are taken to alter their body image}. \cite[p. 1]{Rafati.2021} supports the hypothesis, that Instagram use is at least one factor that increases the dissatisfaction especially beneath women.

A second bigger issue I want to refer to is content moderation. With the beginning of internet platforms, crime also found its way to use these platforms \autocite[][]{Gupta.2021}. Moreover, content that might not be suitable for certain members of social networks (such as pictures that show violence) \autocite[][pp. 225 - 226]{Mengu.2015}, copyright infringements \autocite[][]{Hegemann.2019} and spam \autocite[][p. 1458]{Jin.2011} are also topics the Social Media companies have to deal with. All these topics do have on attribute in common: They do not threat the main revenue streams and the user attention on the platform in their roots. Therefore, only limited amount is spent on solving these issues. A solution to minimize all these negative phenomena is content moderation. On the one hand you are able to win more security and higher content quality by applying such moderation, on the other hand, it is no more than censorship and contradicts the civil right of free speech \autocite[][p. 1359]{Langvardt.2017}. \cite{Langvardt.2017} refers to this indissoluble conflict as \qq{the dilemma of the moderators}. Though there might never be a perfect solution for this difficult topic, in my opinion it is crucial that social media companies - such as Instagram - spend more time and effort within projects and investigation on how to solve these problems. Pure automation will not solve the issue \autocite[][pp. 3 - 4]{Gillespie.2020}.


\section*{Quantification of Social Media Use}
In the second part of this paper, I will quantify my personal Social Network use. For this purpose, I will analyze my visible interaction within the time frame from September 1st until November 30th, 2021. In this time frame I posted one image to my main feed, I shared 32 pictures within the story or the close-friends-story feature, I pressed 108 times the like button, chatted with 36 different persons, from whom I would count 6 as strong ties and placed one comment, within the total use of Instagram, only six comments in total are presented by the report since I registered in 2013. Regarding the messages there should be a further differentiation: Within 13 chats, the conversation consists only out of one to three single messages that often refer to a Instagram story or a content of a third user, that has been shared. On average, Instagram reports 77 viewed posts per day \autocite[][]{Instagram.2021}. 

What might be derived of this data is, that on the one hand, I use Instagram heavily to entertain myself and like the most attractive contents. Personal communication in the sense of reciprocal activity does take place, but does not count up for as much time as the singular activity. Furthermore, it is interesting, that commenting does not have any role for my personal use of the social network whereas messaging is an important and regularly used feature. With strong ties, the communication is much more reciprocal and especially conversation take longer time and have a deeper emotional aspect, whereas communication with weak ties usually consists in likes and quick reactions on their stories.

Regarding the social capital on Instagram, one can differentiate between bonding, bridging and maintaining. While Instagram helps me to keep up bonds with strong ties in the sense that emotional experiences can be shared, it also enhances the possibility to have bridges to certain communities, such as community of my native home village. Though, it has to be mentioned that the effect of bridging weak ties does way more apply on LinkedIn for myself. Maintaining relationships is one of the main reasons to stay on the platform: As already stated, lots of my friends and acquaintances are using this platform and therefore it is an easily usable tool to keep track of what they do, how they feel and what is important in their lives. Therefore, I can totally agree to the statement from the lecture, that strong ties bring reliability to a social network and weak ties expand the scope of the utility of a network: Strong ties i.e., my closest friends convinced me in 2013 to join Instagram, the connection to a lot of weak ties keeps me using it.


\noindent \textbf{Words:} xxxx

\printbibliography

\newpage






\end{document}
