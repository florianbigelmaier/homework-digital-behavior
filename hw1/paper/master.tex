%Doc config

\documentclass[11pt,letterpaper]{article}

\usepackage{graphicx}
\usepackage{textcmds}
\usepackage[left=3cm,right=3cm,top=3cm,bottom=4cm]{geometry}


\usepackage[
    backend=biber,
    style=apa,
    sortlocale=de_DE,
    natbib=true,
    url=false, 
    doi=true,
    eprint=false
]{biblatex}






%\addbibresource{biblio.bib}

\newcommand{\fontsmall}{\fontsize{7pt}{10pt}\selectfont}
\newcommand{\fontnormal}{\fontsize{11pt}{16pt}\selectfont}

\bibliography{biblio}


\begin{document}


    



\noindent\begin{minipage}{0.5\textwidth}
\fontsmall
Management Center Innsbruck \\
Department Management, Communication \& IT  \\
ILV Digital Behavior


\end{minipage}%
\hfill%
\begin{minipage}{0.3\textwidth}\raggedleft
\includegraphics[height=1.0cm]{mci-logo.png}

\end{minipage}


 
\begin{center}
\textbf{\large Homework: Digital Behavior}\\   %Title
Florian Bigelmaier\\                         %Name
\end{center}
\rule{\linewidth}{0.1mm}




\begin{abstract}
    \noindent
Abstract
\end{abstract}

\subsection*{The impact of smartphone use and a personal experiment}
Five hourse every day. This is the amount of time my smartphone reports me as an usual daily display-on time. There is no other interpretation than to asume, that this small device is infleucing my personal life in an tremendous way. As I try to show with some concepts and examples - there are positive aspects and reasons to behave like this. But parallely, being influenced by information technology at such an high level (actually five hours fill almost a third of my active day) has also negative consequences. My personal aim for doing this homework is to come to more conscious decisions and understand my smartphone like a technical instrument, not as an proxy for personal contact as it might appears to be sometimes. As I want to try to reflect on this aim and the process to it, I want to try to formulate this homework in a way that is at least inspired by scientific methodology and standards. Therefore I will switch for the rest of this paper in an objective third person style. When trying to work scientifically it is basically necessary to define a hypothesis that can be validated or neglected. For this paper the hypothesis will be:

\begin{center}
\textit{The personal satisfaction in life increases when the use of the smartphone is limited to conscious usage.}
 \end{center}

\subsection*{Methodology}
This paper does not follow the standards of scientific methodologies as it is supposed to be a personal reflection. Nevertheless this paper follows a strucuted idea of proving the above mentioned hypothesis.

The evaluation of this hypothesis will be measured by a personal experiment with the duration of six days in total. From October, 25th until October, 27th the person will follow the decisions that will be made in a later paragraph. From 28th until 30th of October the person will go back to their actual habits. The person will record their personal experience in a diary. 

The dependent variable of the hypothesis is the \textbf{personal satisfaction}. The personal satisfaction is apparently a complex dimension to be measured. In this case it will be tracked by a diary and the subjective impression how conscious decisions regarding the smartphone use affected the feeling of the person in any related manner. Therefore the above mentioned diary contains for each day notes on the following questions:
\begin{enumerate}
\item 
How long and in which manner did you use the smartphone today?

\item
What are your subjective positive and negative thoughts on the use of the smartphone today?

\item
Which non-digital activities were possible due to the smaller amount of smartphone use? (Especially for the first interval)

\item
Which digital acitivities did you consciously decide not to do?

\item
How did these conscious decisions (regarding the last two questions) affect your personal satisfaction?

\item
Do you want to add any further personal notes regarding your smartphone use today?

\end{enumerate}

The independent variable that will be manipulated during the experiment are named in the hypothesis as \textbf{conscious usage}. More precisely, the affective decisions of smartphone use will be replaced by a rule based decision making for the first above mentioned interval. The rules will emerge out of an analysis of smartphone use and thoughts and concepts that are presented in the following paragraphs. Therefore they will be formulated later on.

\subsection*{Context and Concepts: 14 years of change}
\begin{center}
\qq{Mobile is eating the world} \autocite[][]{evans14}
 \end{center}
When Benedict Evans, a former analyst of the U.S. venture capital fonds Andreesen Horrowitz, formulated this statement in 2014 a revolution just had taken place in the seven years before: In 2007 the first iPhone was released and changed the way how people interact with digital systems dramatically without any question. But beneath this, the iPhone and Smartphone brought several aspects with them, that changed even the every day life for their users - i.e. a big part of the worlds population. Just to name two of them: 
\begin{itemize}
\item 
The \textbf{ubiquity} of the mobile tech architecure including smartphones and the cellular access to the internet made it possible to access the internet wherever a demand could exist \autocite[][p.1]{okazaki13}. This is the basis for situations like paying at the grocery store with digital payment method or using the smartphone as a key for a vehicle.
\item
The \textbf{context sensitivity} of mobile devices differentiates them from former devices: Smartphones are not only small computers, they also come with several sensors that allow e.g. to determin the phones location (via GPS)\autocite[][p.1]{minch04}. Other players in the mobile eco system like providers of apps have the possibility to understand the context of the user by using these sensor data and generate a contextualised environment. An example for this might be the localisation of search results in the Google Maps App. When searching for a restaurant, the App will recommend restaurants that are near to the user and are open for guests at the immediate context of the user.
\end{itemize}

Beneath these new opportunities for entrepreneurs to build new businesses and the end users who profit with an increase of convenience, there is also a shady side of this development. In his essay \qq{Is Google Making Us Stupid?}, \cite[][]{carr08} provocately lists several artefacts of this shady side. For this homework, the most important points of this article are the following ones. All of them with no further references or proven empirical validation.
\cite[][]{carr08} points out, that due to several factors, reading online ist faster and more on the surface of the text not least because of a shorter attention span resulting of the big variety in alternatives for information and entertainment that are only one mouse click away. Furthermore he draws an comparison to the invention of the printing machine and the upcoming revolution of cheaper books. The advantages like more education and availability of information for a broader part of the society and the possibility to publish a bigger variety of knowledge came along with disadvantages. Just like the pros, the cons also are comperable with the digital revolution: The lower the burdens are to publish, the lower the trust is in integrity and authentizity of these publications. A third major point which will come up later again in this homework is the brain change, \cite[][]{carr08} refers to. This change is not a kind of neurological mutation. A change can be more likely observated in the orientation of the individuals. When clocks were available, the way how to plan a day shifted from a sun-oriented and gut-feeling-inspired one to a hard orientation after the 24 hours of the day. \cite[][]{carr08} sees this point in the last 14 years as well: In a more abstract way, smartphones changed the way, individuals orient themselves in their construction of their environment.

An alternative point of view is deliverd by \cite{gergen02} and \cite{ward17} who are referring to the concepts of absent presence respectively the cognitive consequences of smartphone use.
\begin{center}
\qq{We are present but simultaneously rendered absent; we have been erased by an absent presence} \autocite[][p.227]{gergen02}
\end{center}
\cite{gergen02} already pointed out in 2002 that the use of modern forms of communication is so immersive, that people do not pay attention to and interact actively with their environment in which they are physically present. \cite{gergen02} is referring to this as a challenge. He states that \qq{The erosion of face-to-face community, a coherent and centered sense of
self, moral bearings, depth of relationship, and the uprooting of meaning from material context [...] are [...] repercussions of absent presence}\autocite[][p.236]{gergen02}. In addition to that, \cite{ward17} researched on the question how the extensive use of smartphones and the integration in the daily live affects the cognitive ressources of humans. In two experiments they could prove, \qq{that the mere presence of consumers’smartphones can adversely affect two measures of cognitive capacity — available
working memory capacity and functional fluid intelligence} \autocite[][]{ward17}. The combined point of view of \cite{gergen02} and \cite{ward17} will be a basis for providing support to the concious decisions to make for the first interval of the experiment.


\subsection*{Analysis of smartphone use}
Row data
categorization
implications

\subsection*{Derivating a plan for the experiment}

\subsection*{Observations during the experiment}

\subsection*{The effect of conscious usage regarding smartphone use on the personal satisfaction}

\subsection*{Discussion}

\newpage





\printbibliography 

\end{document}
