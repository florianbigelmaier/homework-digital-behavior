%Doc config

\documentclass[11pt,letterpaper]{article}

\usepackage[
backend=biber,
style=alphabetic,
sorting=ynt
]{biblatex}
\addbibresource{biblio.bib}


\setlength{\parindent}{0em}                  %Einrückungsabstand
\setlength{\parskip}{0.5em}                  %ABSTAND ZWISCHEN DEN ABSÄTZEN
\textwidth 6.5in
\textheight 9.in
\oddsidemargin 0in
\headheight 0in

%Module

\usepackage{textcmds}
\usepackage{fancybox}
\usepackage[utf8]{inputenc}
\usepackage{epsfig,graphicx}
\usepackage{multicol,pst-plot}
\usepackage{pstricks}
\usepackage{amsmath}
\usepackage{amsfonts}
\usepackage{amssymb}
\usepackage{eucal}
\usepackage[left=2cm,right=2cm,top=2cm,bottom=2cm]{geometry}
\usepackage{txfonts}
\usepackage[english]{babel}
\usepackage[colorlinks]{hyperref}
\usepackage{cancel}
\usepackage{caption}
\usepackage{float}
\usepackage{upgreek}
\usepackage{gensymb}
\usepackage{subfigure}
\usepackage{siunitx}
\usepackage{color}
%\usepackage{tikz}
\usepackage{listings}
%\usepackage{minted}
%\usepackage{mdframed}
%\usepackage{natbib}
%\bibliographystyle{mnras}

%\setcitestyle{aysep{","}}
\usepackage{multicol}

\renewcommand{\bibpreamble}{\begin{multicols}{2}}
\renewcommand{\bibpostamble}{\end{multicols}}
\setlength{\bibsep}{3pt}

%colors

\definecolor{codegreen}{rgb}{0,0.6,0}
\definecolor{codegray}{rgb}{0.5,0.5,0.5}
\definecolor{backcolour}{rgb}{0.95,0.95,0.95}
\hypersetup{colorlinks=true,linkcolor=codegreen,citecolor=blue,filecolor=blue,urlcolor=magenta,}

%configuration listings

\lstset{ %
language=python,                % choose the language of the code
basicstyle=\footnotesize,       % the size of the fonts that are used for the code
numbers=left,                   % where to put the line-numbers
numberstyle=\footnotesize,      % the size of the fonts that are used for the line-numbers
stepnumber=1,                   % the step between two line-numbers. If it is 1 each line will be numbered
numbersep=5pt,                  % how far the line-numbers are from the code
backgroundcolor=\color{white},  % choose the background color. You must add \usepackage{color}
showspaces=false,               % show spaces adding particular underscores
showstringspaces=false,         % underline spaces within strings
showtabs=false,                 % show tabs within strings adding particular underscores
frame=single,                   % adds a frame around the code
tabsize=2,                      % sets default tabsize to 2 spaces
captionpos=b,                   % sets the caption-position to bottom
breaklines=true,                % sets automatic line breaking
breakatwhitespace=false,        % sets if automatic breaks should only happen at whitespace
escapeinside={\%*}{*)}          % if you want to add a comment within your code
}
\lstdefinestyle{mystyle}{
	backgroundcolor=\color{backcolour},   
	commentstyle=\color{red},
	keywordstyle=\bfseries\color{magenta},
	numberstyle=\tiny\color{codegray},
	stringstyle=\color{codegreen},
	basicstyle=\footnotesize\ttfamily,
	identifierstyle=\color{blue},
	breakatwhitespace=false,         
	breaklines=true,                 
	captionpos=b,                    
	keepspaces=true,                 
	numbers=left,                    
	numbersep=5pt,                  
	showspaces=false,                
	showstringspaces=false,
	showtabs=false,                  
	tabsize=2
}

\lstset{style=mystyle}

%configuration minted style

%\usemintedstyle{vs}

%extra comands




%COMIENZA EL DOCUMENTO

\begin{document}



%CONFIGURACIÓN DEL ENCABEZADO

\usetikzlibrary{positioning}
\tikzset{every picture/.style={line width=0.75pt}}    
\pagestyle{plain}
%\begin{flushleft}
%Department Management, Communication \& IT  \\
%ILV Digital Behavior\\
%\underline{Management Center Innsbruck} \hfill   \includegraphics[height=1.0cm]{mci-logo.png}
%\end{flushleft}


\noindent\begin{minipage}{0.5\textwidth}% adapt widths of minipages to your needs
Department Management, Communication \& IT  \\
ILV Digital Behavior\\
\underline{Management Center Innsbruck} 

\end{minipage}%
\hfill%
\begin{minipage}{0.3\textwidth}\raggedleft
\includegraphics[height=1.0cm]{mci-logo.png}

\end{minipage}

%\begin{flushright}\vspace{-5mm}
%\includegraphics[height=1.0cm]{mci-logo.png}
%\end{flushright}
 
\begin{center}\vspace{-1cm}
\textbf{\large Homework: Digital Behavior}\\   %Title
Florian Bigelmaier\\                         %Name
\end{center}
\rule{\linewidth}{0.1mm}


%DESDE AQUÍ SE ESCRIBE TODO EL CONTENIDO

\begin{abstract}
    \noindent
    Esta template de \LaTeX viene preparada con muchos paquetes útiles, ya sea para escribir resoluciones matemáticas, importar imágenes, figuras, códigos, crear hipervínculos, signos matemáticos y mucho más. La he preparado durante mis últimos 2 años en la universidad, para poder entregar trabajos ordenados y completos. Ha sido probar muchos paquetes, ver errores, solucionarlos, editar y personalizar estilos hasta al fin encontrar algo que me guste y poder compartir con los demás para que puedan ocuparlo directamente o tener una base bien estructurada para poder crear sus propias templates, espero sea de utilidad para cualquiera que llegue hasta acá\footnote{Última edición: 27 de Agosto, 12:54}.
\end{abstract}

\subsection*{Goal and scope of this homework}
\begin{center}
\qq{Mobile is eating the world} cite(evans14)
 \end{center}
 
When Benedict Evans, a former analyst of the U.S. venture capital fonds Andreesen Horrowitz, formulated this statement in 2014 a revolution just had taken place in the seven years before: In 2007 the first iPhone was released and changed the way how people interact with digital systems dramatically without any question. But beneath this, the iPhone and Smartphone brought several aspects with them, that changed even the every day life for their users - i.e. a big part of the worlds population. Just to name three of them: The ubiquity of the mobile tech architecure including smartphones and the cellular access to the internet made it possible to access the internet wherever a demand could exist. This is the basis for situations like paying at the grocery store with digital payment method or navigating with the smartphone through the streets of a big city with an real time optimization for the fastest route.

\subsection*{Comandos personalizados}

Hay par de comandos personalizados que están un poco más arriba en el código (pienso incluir varios), y que ayudan con operadores en mecánica cuántica, astronomía y cálculo.\par

Es cierto que algunos comandos vienen ya en otros paquetes, sin embargo, quería que esta template tuviera sólo lo necesario y no un exceso de comandos que jamás se usaran, por eso a medida que voy necesitando nuevos comandos, yo mismo los voy creando como comando personalizado. Aquí algunos ejemplos de operadores bra y ket de mecánica cuántica:

Ejemplos de comando unidad y unidades de medida astronómicas:


Y finalmente ejemplos del comando probabilidad, valor absoluto y evaluar integral:




\newpage
\bibliography{biblio}






\end{document}
